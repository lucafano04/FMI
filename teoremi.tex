\documentclass{article}
\usepackage[italian]{babel}
\usepackage[utf8]{inputenc}
\usepackage{amsmath}
\usepackage{amsthm}
\usepackage{amsfonts}
\usepackage{amssymb}
\usepackage{cancel}
\usepackage[margin=1in]{geometry}

\makeatletter
\newcommand\subsubparagraph{\@startsection{paragraph}{4}{\z@}%
  {-3.25ex\@plus -1ex \@minus -.2ex}%
  {1.5ex \@plus .2ex}%
  {\normalfont\normalsize\bfseries}}
\makeatother

\makeatletter
\renewenvironment{proof}[1][\proofname]{\par
    \pushQED{\qed}%
    \normalfont \topsep6\p@\@plus6\p@\relax
    \trivlist
    \item\relax
    {\itshape
    #1\@addpunct{.}}\hspace\labelsep\ignorespaces
    }{%
    \popQED\endtrivlist\@endpefalse
}
\makeatother

\newtheorem{theorem}{Teorema}[part]
\newtheorem{lemma}{Lemma}[theorem]
\newtheorem{corollary}[lemma]{Corollario}
\newtheorem{propsosition}[lemma]{Proposizione}
\newtheorem{definition}[lemma]{Definizione}
\newtheorem{ipothesis}[lemma]{Ipotesi}
\newtheorem{thesis}[lemma]{Tesi}
\newtheorem{demonstration}[lemma]{Dimostrazione}
\newtheorem*{lemma*}{Lemma}

% defnition of existence and uniqueness and also induction base and inductive step 
\theoremstyle{definition}
\newtheorem*{existence}{Esistenza}
\newtheorem*{uniqueness}{Unicità}
\newtheorem*{base}{Base Induttiva}
\newtheorem*{step}{Passo Induttivo}

% Definition of util macros
\newcommand{\Z}{\mathbb{Z}}
\newcommand{\N}{\mathbb{N}}
\newcommand{\Q}{\mathbb{Q}}
\newcommand{\R}{\mathbb{R}}


\title{Definizioni ed dimostrazioni di teoremi per il corso di Fondamenti Matematici per l'Informatica}
\author{Facchini Luca}
\date{A.A. 2023/24}

\begin{document}
\maketitle
\begin{abstract}
    In questo documento sono presenti le definizioni e le dimostrazioni dei teoremi richiesti dal Prof. Ghiloni R. per l'esame del corso di Fondamenti Matematici per l'Informatica dell'anno accademico 2023/24.
\end{abstract}
\tableofcontents
\pagebreak
\part[Ordinamento numeri naturali e seconda forma induzione]{L’ordinamento dei numeri naturali è un buon ordinamento e seconda forma del principio di induzione}
    \section{Ordinamento dei numeri naturali}
        \begin{theorem}
            L'insieme dei numeri naturali $\mathbb{N}$ è un insieme ordinato rispetto alla relazione d'ordine $\leq$.
            \begin{ipothesis}
                Sia \(A \subseteq \mathbb{N}\).
            \end{ipothesis}
            \begin{thesis}
                se \(A \subseteq \mathbb{N}\) non ha un minimo allora \(A=\emptyset\), dunque \(\mathbb{N}\) è un insieme ben ordinato.
            \end{thesis}
            \begin{proof}
                Definiamo con \(B\) il complementare di \(A\), (\(B:=A^C=\mathbb{N}\setminus A\)), verifichiamo l'ipotesi per induzione di prima forma su \(B\), dunque che \(\{0,1,\ldots,n\}\in B\ \forall n\in\mathbb{N}\Rightarrow A = \emptyset\).
                Assumiamo che \(\mathbb{N}\) non sia ben ordinato e che dunque se \(A\neq\emptyset\) allora \(A\) non ha un minimo.
                \begin{base}[$0\in B$]
                    Dunque \(0\notin A\) altrimenti questo ne sarebbe il minimo, dunque \(\{0\}\subseteq B\) in quanto \(B\) è definito come il complementare di \(A\).\checkmark
                \end{base}
                \begin{step}[$n\in B\Rightarrow n+1\in B$]
                    Supponendo ora che \(0,1,\ldots,n\in B\), allora \(0,1,\ldots,n\notin A\), ciò implica che \(n+1\in A\) questo però lo renderebbe un minimo il che è inammisibile in quanto \(A\) non ha un minimo per ipotesi. Dunque \(n+1\in B\), e quindi \(\{0,1,\ldots,n,n+1\}\subseteq B\). Il passo induttivo è verificato.\checkmark
                \end{step}
                Dunque per induzione di prima forma su \(B\) abbiamo che \(A=\emptyset\), e quindi \(\mathbb{N}\) è un insieme ben ordinato.
                \pushQED{}
            \end{proof}
            \raggedleft{\ensuremath{\blacksquare}}
        \end{theorem}
    \section{Seconda forma del principio di induzione}
        \begin{theorem}
            Seconda forma del principio di induzione:
            \begin{ipothesis}
                Sia \(P(n)\) una serie di proposizioni indicizzata su \(\mathbb{N}\), assumendo che:
                \begin{enumerate}
                    \item \(P(0)\) sia vera.
                    \item \(\forall n\in\mathbb{N}\) se \(P(0),P(1),\ldots,P(n)\) sono vere.
                \end{enumerate}
            \end{ipothesis}
            \begin{thesis}
                Se le ipotesi sono verificate allora \(P(n)\) è vera \(\forall n\in\mathbb{N}\).
            \end{thesis}
            \begin{proof}
                Sia \(A:=\{n\in\mathbb{N}\mid P(n)\text{ è falsa}\}\), dimostriamo che \(A=\emptyset\). Supponendo che \(A\neq\emptyset\Rightarrow \exists n\in\mathbb{N}: n=\min (A)\).
                \begin{base}[$P(0)$]
                    Quindi abbiamo che esiste un minimo su \(A\), per la bese induttiva (1) \(0\notin A\) dunque \(0\neq\min (A)\) in quanto \(P(0)\) è vera.
                \end{base}
                \begin{step}[$P(n)\Rightarrow P(n+1)$]
                    Inoltre se \(k<n\quad k\notin A\) in quanto \(n=\min(A)\) e dunque \(P(k)\) è vera, a questo punto per il passo induttivo (2) abbiamo che \(\forall k<n\quad P(k)\text{ è vera}\) e dunque \(P(n)\) è vera, il che và in contraddizione con la definizione dell'esistenza di un minimo su \(A\). Dunque \(A=\emptyset\), e quindi \(P(n)\) è vera \(\forall n\in\mathbb{N}\).
                \end{step}
            \pushQED{}
            \end{proof}
            \raggedleft{\ensuremath{\blacksquare}}
        \end{theorem}
\pagebreak
\part{Teorema dell'esistenza e dell'unicità del quoziente e del resto della divisione euclidea}
\section{Unicità e esistenza del quoziente e del resto della divisione euclidea}
\begin{theorem}
    Il quoziente ed il resto della divisione euclidea di un numero naturale \(n\) per un numero naturale \(m\neq 0\) esistono ed sono unici.
    \begin{ipothesis}
        Siano \(n,m\in\mathbb{N}\) con \(b\neq 0\).
    \end{ipothesis}
    \begin{thesis}
        Esistono ed sono unici due numeri naturali \(q,r\in\mathbb{N}\) tali che \(\begin{cases} n=m\cdot q+r\\0\leq r<m\end{cases}\).
    \end{thesis}
    \begin{proof}
        \begin{existence}
            Si procede per induzione di seconda forma su \(n\) partendo da \(n=0\).
            \begin{base}[$n=0$]
                Per \(n=0\) si ha che \(0=0\cdot m+0\), dunque \(q=0\) e \(r=0\) allora \(\forall m\in\mathbb{Z}\ P(0)\) è verificate.
            \end{base}
            \begin{step}[$\forall k<n\ P(k)\Rightarrow P(n)$]
                Per \(n\geq1\) si supponga che \(\forall k<n\ P(k)\) sia verificata, dobbiamo verificare l'esistenza del quoziente e del resto della divisione euclidea di \(n\) per \(m\).
                \begin{list}{\leftmargin=1.5em}
                    \item{\(n<m\land n\geq 0\)} allora \(n=m\cdot 0+n\), dunque \(q=0\) e \(r=n\), e quindi \(\forall n<m\ P(n)\) è verificata.
                    \item{\(n\geq m\land m>0\)} allora poniamo \(k:=n-m\), ovvervando che \(0\leq k<n\) allora per ipotesi induttiva \(\exists q,r\in\N\) t.c. \(\left<\begin{aligned}k=m\cdot q+r\\0\leq r<(n-m)\end{aligned}\right.\), vale quindi: \(n=k+m=(qm+r)+m=(q+1)m+r\) quindi il sistema è scrivibile come: \(\begin{cases}n=m\cdot (q+1)+r\\0\leq r<m\end{cases}\), quindi il passo induttivo è stato fatto, e grazie al principio di induzione di 2° forma \(\exists\) sempre \(q,r\) della divisione eucludea di \(n\) per \(m\) con \(n,m\in\N\quad m> 0\).
                    \item{\(n<0\land m>0\)} Grazie alla dimostrazione precedente, \(\exists q,r\in\N\) t.c. \(\begin{cases}n=-m\cdot q+r\\0\leq r<m\end{cases}\), Vale: \(-n=qm+r\Rightarrow n=-qm-r=(-q)m-r\). Se \(r=0\) \(n=(-q)m\), Supponendo che \(r>0\) ovvero \(0<r<m\Leftrightarrow 0<m-r<m\) vale: \(n=(-q)m-r=(-q)m-m+m-r=(-q-1)m+(m-r)\), quindi: \(-q-1\) è il quoziente e \(m-r\) è il resto, dunque \(\forall n<0\ P(n)\) è verificata.
                    \item{\(m<0\)} allora \(-m>0\) dunque \(\exists q,r \in\mathbb{N}\) tali che \(n=-m\cdot q+r\) con \(0\leq r<-m\), dunque \(n=m\cdot q-r\), e quindi \(q=-q\) e \(r=-r\), e quindi \(\forall m<0\ P(n)\) è verificata.
                \end{list}
            \end{step}
            Il passo induttivo è verificato, dunque per induzione di seconda forma su \(n\) abbiamo che esistono il quoziente ed il resto della divisione euclidea di \(n\) per \(m\).
        \end{existence} 
        \qed
        \begin{uniqueness}
            Proviamo che \(q=q', r=r'\):
            Se \(r'>r\) a meno di riordinamento, allora vale che: \(qm-q'm=r'-r\Leftrightarrow m(q-q')=r'-r\).
            Effettuando l'operazione al modulo otteniamo: \(|m(q-q')|=|r'-r|\), dunque \(|r'-r|<m\), questo è vero se e solo se \(0\leq|q-q'|<1\),
            Questo implica che \(q=q'\) in quanto \(q,q'\in\mathbb{N}\).
            Quindi \(mqr=mq'+r'\Rightarrow q=q'\) e \(r=r'\).
            Quindi il quoziente ed il resto della divisione euclidea di \(n\) per \(m\) sono unici.
        \end{uniqueness}
        \raggedleft{\pushQED{\ensuremath{\blacksquare}}}
    \end{proof}
\end{theorem}
\part{Teorema di esistenza e unicità della rappresentazione in \(n\) base \(\ge 2\)}
\section{Esistenza e unicità della rappresentazione in base \(b\ge 2\)}
\begin{theorem}
    Un numero naturale \(n\) può essere sempre rappresentato in base \(b\ge 2\) in modo unico.
    \begin{ipothesis}
        Siano \(n,b\in\mathbb{N}\) con \(b\ge 2\).
    \end{ipothesis}
    \begin{thesis}
        Esistono ed sono unici (\(\exists!\)) una rappresentazione di \(n\) in base \(b\), ovvero una successione \(\{\varepsilon_i\}\) con le seguenti proprietà:
        \begin{enumerate}
            \item \(\varepsilon_{i\in\mathbb{N}}\) definitivamente nulla, ovvero dopo qualche \(i_0\in\mathbb{N}\Rightarrow \forall j>i_0: \varepsilon_j=0\).
            \item \(\varepsilon_i\in I_b:=\{0,1,\ldots,b-1\}\quad \forall i\in\mathbb{N}\quad (0\leq\varepsilon_i< b)\).
            \item \(\displaystyle\sum_{i\in\mathbb{N}} \varepsilon_i\cdot b^i=n\).
        \end{enumerate}
        Inoltre se esiste \(\{\varepsilon'_i\}_{i\in\mathbb{N}}\) rappresentazione di \(n\) in base \(b\) allora \(\varepsilon_i=\varepsilon'_i\quad \forall i\in\mathbb{N}\).
    \end{thesis}
    \begin{proof}
        %\textbf{Esistenza:} 
        \begin{existence}
            Si procede per induzione di seconda forma su \(n\) partendo da \(n=0\).
            \begin{base}[$n=0$] 
                Per \(n=0\) si ha che \(0=0\cdot b^0\), dunque \(\varepsilon_0=0\) e quindi \(\forall n\in\mathbb{N}\ P(n)\) è verificata.\end{base}
            \begin{step}[$\forall k<n\ P(k)\Rightarrow P(n)$]
                Per \(n\geq 1\) si supponga che \(\forall k<n\ P(k)\) sia verificata, e dunque che esista una rappresentazione di \(k\) in base \(b\). Eseguiamo la divisione euclidea di \(n\) per \(b\), dunque \(\exists q,r\in\mathbb{N}\) tali che \(n=b\cdot q+r\) con \(0\leq r<b\), per ipotesi \(b\geq 2\) dunque \(0<q<qb\leq qb+r=n\),
                per ipotesi induttiva è vero che \(q\) è rappresentabile come una successione \(\{\delta_i\}_{i\in\mathbb{N}}\) con le proprietà (1),(2),(3),
                inoltre vale che:
                \[
                    \begin{aligned}
                        n&=(\sum_{i\in\mathbb{N}}\delta_ib^i)b+n\Rightarrow n=\sum_{i\in\mathbb{N}}\delta_ib^{i+1}+r\quad \text{Definiamo: }\epsilon_0&=r\\
                        n&=\epsilon_0+\sum_{j\geq 1}\delta_{j-1}b^j=\sum_{i\in\mathbb{N}}\epsilon_ib^i
                    \end{aligned}
                \]
            \end{step}
        \qed
        \end{existence}
        \begin{uniqueness}
            Procediamo per induzione di seconda forma su \(n\) da \(n=0\)
            \begin{base}[$n=0$]
                Per \(n=0\) \(\epsilon_i=0\quad \forall i\in\mathbb{N}\), questa è l'unica rappresentazione di \(0\) in base \(b\).
            \end{base}
            \begin{step}[$\forall k<n\ P(k)\Rightarrow P(n)$]
                Assumendo che esistano \(\{\epsilon_i\}_{i\in\mathbb{N}}\quad \{\epsilon'_i\}_{i\in\mathbb{N}}\) coin le proprietà (1),(2),(3), proviamo che \(\epsilon_i=\epsilon'_i\quad \forall i\in\mathbb{N}\). 
                Dalla dimostrazine precendete osserviamo:
                \[
                    \begin{aligned}
                        n&=\sum_{i\in\mathbb{N}}\epsilon_ib^i=\sum_{i\in\mathbb{N}}\epsilon'_ib^i\\
                        \Rightarrow \epsilon_0+b(\sum_{i\geq1}\epsilon_ib^{i-1})&=\epsilon'_0+b(\sum_{i\geq1}\epsilon'_ib^{i-1})
                    \end{aligned}
                \]
                Dove \(\epsilon_0\) e \(\epsilon'_0\) sono i resti della divisione euclidea di \(n\) per \(b\), in quanto questi uguali in entrambi i casi per il torema dell'unicità del quoziente e del resto questi sono uguali.
                Inoltre dato che \(\sum_{i\geq1}\epsilon_ib^{i-1}=\sum_{i\geq1}\epsilon'_ib^{i-1}\) per ipotesi, dato che sono \(<n\) allora la loro rappresentazione è unica quindi \(\forall i>1\ \epsilon_i=\epsilon'_i\), unendo, otteniamo che \(\epsilon_i=\epsilon'_i\quad \forall i\in\mathbb{N}\).
                Il passo induttivo è stato fatto e l'unicità della rappresentazione è stata dimostrata.
            \end{step}
        \end{uniqueness}
        \raggedleft{\pushQED{\ensuremath{\blacksquare}}}
    \end{proof}
\end{theorem}
\part{Teorema di esistenza e unicità del massimo comune divisore e del minimo comune multiplo}
\section{Massimo comune divisore}
\begin{theorem}
    Il massimo comune divisore tra due numeri \(n\) e \(m\) esiste ed è unico.
    \begin{ipothesis}
        Siano \(n,m\in\mathbb{Z}\) con \(n,m\) non entrambi nulli.
    \end{ipothesis}
    \begin{thesis}
        Esiste \(\exists d\) che è MCD di \(n,m\) se:
        \begin{enumerate}
            \item \(d|n\) e \(d|m\).
            \item se \(c|n\) e \(c|m\) allora \(\Rightarrow c|d\).
        \end{enumerate} 
        Inoltre: Se \(\exists\) M.C.D tra \(n,m\) allora questo è unico e lo indichiamo con \((n,m)\).
        \begin{lemma}[Lemma utile]
            \(d\) è espremibile come combinazione lineare di \(n\) e \(m\).
            \[
                \exists x,y\in\mathbb{Z}: d=nx+my
            \]
        \end{lemma}
    \end{thesis}
    \begin{proof}
        \begin{uniqueness}
            Supponendo che \(\exists d_1,d_2\in\mathbb{N}\) che rispettino (1) e (2). Applichiamo allora queste ottenedo:
            \[
                \begin{aligned}
                    (1)&\ d_1|n \wedge d_1|m\\
                    (2)&\ c=d_1\ d_1|n \wedge d_1|m \Rightarrow d_1|d_2
                \end{aligned}
            \]
            applicando l'inverso si ottiene che \(d_2|d_1\), dunque \(d_1=\pm d_2\), ma dato che \(d_1,d_2\in\mathbb{N}\) allora \(d_1=d_2\), e quindi il M.C.D è unico.
            \qed
        \end{uniqueness}
        \begin{existence}
            Sia \(S:=\{nx+my\mid x,y\in\mathbb{Z}\}\), definito come l'insieme delle combinazioni lineari di \(n\) e \(m\), questo insieme è non vuoto in quanto \(nn+mm>0\in S\).
            Esiste dunque un minimo elemento in \(S\), chiamiamolo \(d=\min S\), vale che: 
            \[
                \begin{aligned}
                    &d|n \wedge d|m\\
                    &\exists c\in\mathbb{Z}: c|n \wedge c|m \Rightarrow c|d
                \end{aligned}
            \]
            in quanto \(d\in S\).
            Dalla proprietà (2) si deduce \[c|xm+ym\]
            Si prova ora che \(d|n\) tramite la divisione euclidea di \(n\) per \(d\), ottendendo dunque \(n=qd+r\), ponendo per assurdo che \(r>0\) allora \(r\in S\) e quindi \(d\neq \min S\) in quanto \(r<d\), assumendo che sia vero:
            \[
                \begin{aligned}
                    r&=n-qd=n-q(xd+ym)=\\
                    &=n-qnx-qmy=\\
                    &=n(1-qn)+m(-qy)\in S
                \end{aligned}
            \]
            dunque è verificato che il resto della divisione euclidea è in \(S\), e quindi che il \(\min S\neq d\) ma per definzione \(d:=\min S\), il che è un assurdo e quindi \(r=0\) il che dimostra che \(d|n\), analogamente si dimostra che \(d|m\).
        \end{existence}
        \raggedleft{\pushQED{\ensuremath{\blacksquare}}}
    \end{proof}
\end{theorem}
\section{Minimo comune multiplo}
\begin{theorem}
    Il minimo comune multiplo tra due numeri \(n\) e \(m\) esiste ed è unico.
    \begin{ipothesis}
        Siano \(n,m\in\mathbb{Z}\)
    \end{ipothesis}
    \begin{thesis}
        \(\exists!M\in\mathbb{N}\) che è m.c.m di \(n\) e \(m\), se:
        \begin{enumerate}
            \item \(n|M\land m|M\).
            \item Se \(n|c \land m|c \Rightarrow M|c\) per qualche \(c\in\mathbb{N}\). 
        \end{enumerate}
        Inoltre: Se \(\exists\) m.c.m tra \(n\) e \(m\) allora questo è unico e lo indichiamo con \([n,m]\), e vale se \(n,m\) non sono entrambi nulli vale che: \([n,m]=\frac{n\cdot m}{(n,m)}\), altrimenti \([n,m]=0\).
    \end{thesis}
    \begin{proof}
        \begin{uniqueness}
            Supponiamo che esistano \(M_1,M_2\in\mathbb{N}\) che rispettino (1) e (2), applicando queste otteniamo:
            \[
                \begin{aligned}
                    (1)&\ M_1|n \land M_1|m\\
                    (2)&\ c=M_1\ c|n \land c|m \Rightarrow M_1|c
                \end{aligned}
            \]
            applicando l'inverso si ottiene che \(M_2|c\), dunque \(M_1=\pm M_2\), ma dato che \(M_1,M_2\in\mathbb{N}\) allora \(M_1=M_2\), e quindi il m.c.m è unico.
            \qed
        \end{uniqueness}
        \begin{existence}
            Supponendo che \(n,m\) non sono entrambi nulli, altrimenti \([n,m]\exists :=0\) allora:
            \[
                \begin{aligned}
                    \Rightarrow (n,m)\mid n&\Leftrightarrow n=n'(n,m)&\text{ per qualche } n' \in\mathbb{Z}\\
                    \Rightarrow (n,m)\mid m&\Leftrightarrow m=m'(n,m)&\text{ per qualche } m' \in\mathbb{Z}
                \end{aligned}
            \]
            Definendo \(M:=\frac{n\cdot m}{(n,m)}\) e sostituendo \(n,m\) otteniamo che:
            \[
                M=\frac{n'm'(n,m)(n,m)}{(n,m)}=n'm'(n,m)
            \]
            ma per per la proprietà associativa della moltiplicazione, e per la definizione precedente di \(n',m'\) otteniamo che: 
            \[
                M=\begin {cases}
                    (n'(n,m))m'&=nm'\\
                    (m'(n,m))n'&=n'm
                \end{cases}\Rightarrow n'm = nm'
            \]
            quindi la proprietà \((1)\) è verificata perchè \(n|M\) e \(m|M\).
            Per verificare la proprietà \((2)\) controlliamo che per \(c\in\mathbb{Z}\) vale che \(n|c\land m|c\Rightarrow M|c?\)
            \[
                \begin{aligned}
                (n,m)|n,n|c\Rightarrow (n,m)|c&\\
                (n,m)|m,m|c\Rightarrow (n,m)|c&
                \end{aligned}\Rightarrow c=c'(n,m)
            \]
            Inoltre per definizione di \(n',m'\Rightarrow(n',m')=1\), dunque \(n'|c'\land m'|c'\Rightarrow n'm'|c'\), moltiplicando l'equazione per \((n,m)\) otteniamo: \(n'm'(n,m)|c'(c,m)\Rightarrow M|c\), dunque la proprietà \((2)\) è verificata e l'esistenza dimostrata.
        \end{existence}
        \raggedleft{\pushQED{\ensuremath{\blacksquare}}}
    \end{proof}
\end{theorem}
\pagebreak
\part{Teorema fondamentale dell'aritmetica}
\section{Teorema di esistenza e unicità della fattorizzazione in numeri primi}
\begin{theorem}
    Ogni numero naturale \(n\geq 2\) è scrivibile come prodotto di numeri primi in modo unico, a meno 
    \begin{ipothesis}
        Sia un numero \(n\in\mathbb{Z}, n\geq 2\).
    \end{ipothesis}
    \begin{thesis}
        Esistono numeri primi \(p_1,p_2,\ldots,p_k>0\) tali che \(n=p_1\cdot p_2\cdot\ldots\cdot p_k\).
        Se anche \(q_1,q_2,\ldots,q_l\) sono numeri primi tali che \(n=q_1\cdot q_2\cdot\ldots\cdot q_l\) allora esiste una bigezzione \(\delta:\{1,2,\ldots,l\}\rightarrow \{1,2,\ldots,k\}\) tale che \(q_i=p_{\delta(i)}\).
    \end{thesis}
    \begin{proof}
        \begin{existence}
            Si procede perinduzione di 2° forma shiftata su \(n\), partendo da \(n=2\).
            Se \(n=2\) è scrivibile come prodotto di numeri primi e ogni numero \(k<n\) è scrivibile come prodotto di numeri primi allora \(n\) è scrivibile come prodotto di numeri primi.
            \begin{base}[$n=2$]
                Per \(n=2\) si ha che \(2=2\), dunque \(2\) è scrivibile come prodotto di numeri primi.
            \end{base}
            \begin{step}[$\forall k<n\ P(k)\Rightarrow P(n)$]
                Per \(n>2\) si supponga che \(\forall k<n\ P(k)\) sia verificata, e dunque che \(k\) sia scrivibile come prodotto di numeri primi, la dimostrazione si suddivide in due casi:
                \begin{itemize}
                    \item Se \(n\) è primo allora \(n\) è scrivibile come prodotto di numeri primi.
                    \item Se \(n\) non è primo allora esstono almento due numeri \(d_1,d_2\) tali che \(1<d_1,d_2<n\quad n=d_1\cdot d_2\). Per ipotesi induttiva in quanto \(d_1,d_2<n\) allora \(d_1,d_2\) sono scrivibili come prodotto di numeri primi: \(d_1=p_1\cdot p_2\cdot\ldots\cdot p_k\) e \(d_2=q_1\cdot q_2\cdot\ldots\cdot q_l\), dunque \(n=p_1\cdot p_2\cdot\ldots\cdot p_k\cdot q_1\cdot q_2\cdot\ldots\cdot q_l\), e quindi \(n\) è scrivibile come prodotto di numeri primi.
                \end{itemize}
            \end{step}
            Il passo induttivo è verificato, dunque per induzione di 2° forma shiftata su \(n\) abbiamo che \(n\) è scrivibile come prodotto di numeri primi.
            \qed
        \end{existence}
        \begin{uniqueness}
            Siano \(n=p_1\cdot p_2\cdot\ldots\cdot p_k=q_1\cdot q_2\cdot\ldots\cdot q_h\) con \(p_i,q_j\) numeri primi, inoltre si supponga che per \(k\leq h\). Si procede per induzione di seconda forma shiftata su \(k\) da \(k=1\).
            \begin{base}[$k=1$]
                Per \(k=1\) si ha che \(n=p_1=q_1\cdot q_2\cdot\ldots\cdot q_h\) dunque \(q_j|p_1\forall i\in \{1,2,\ldots,h\}\), il che è possibile ma \(\Leftrightarrow q_j=\pm 1\lor q_j=\pm p_1\), in quanto però abbioamo definito per ipotesi che \(q_j>1\) allora l'unica possibilità è che \(q_j=p_1\), ma se \(h>1\) allora \(q_j=p_1\quad \forall j\in\{1,2,\ldots,h\}\), che comporta che se \(h>k\Rightarrow n=q_1\ldots q_h\geq q_1q_2\Rightarrow p_1^2>p_1\) in quanto \(p_1>1\), il che è un assurdo, dunque \(h=k=1\), e quindi \(n=p_1=q_1\).
            \end{base}
            \begin{step}[\(\forall h<k\ P(h)\Rightarrow P(k)\)]
                Sia ora \(k>1\) allora \(p_k|n=q_1\ldots q_n\) dunque in quanto \(q_1,\dots,q_n\) sono primi \(\exists\) almeno un \(p_k|q_j\) allora \(p_k=q_j\) in quanto come detto precedentemente \(p_k,q_k\text{ primi}>1\). Seguendo la divisione euclidea \(p_1\cdot\ldots p_{k-1} = q_1\cdot\ldots\cdot q_{j-1}\cdot q_{j+1}\cdot\ldots\cdot q_n\), questi sono numeri \(<n\) per ipotesi induttiva dunque hanno lo stesso numero di elementi: \(k-1=h-1\) e che eiste una bigezzione \(\delta:\{1,2,\ldots,h-1\}\rightarrow\{1,2,\ldots,k-1\}\), possiamo ora definire una bigezzione \(\sigma:\{1,2,\ldots,h\}\rightarrow\{1,2,\ldots,k\}\) tale che:\[
                    \sigma(i):\begin{cases}
                        \delta(i)&\text{ se } i\neq j\\
                        k&\text{ se } i=j
                    \end{cases}
                \] 
                dunque è stata definita una bigezzione tale che \(q_i=p_{\sigma(i)}\quad \forall i\in\{1,2,\ldots,h\}\).
            \end{step}
        \end{uniqueness}
        Il passo induttivo è verificato, dunque per induzione di seconda forma shiftata su \(k\) abbiamo che se \(n\) è scrivibile come prodotto di numeri primi allora questa è unica a meno di riordinamento in quanto esiste una bigezione tra gli indici delle sequenze di numeri primi.
        {\raggedleft{\pushQED{\ensuremath{\blacksquare}}}}
    \end{proof}
\end{theorem}
\part{Teorema cinese del resto}
\section{Teorema cinese del resto}
\begin{theorem}
    \begin{ipothesis}
        Siano \(a,b,n,m\in\mathbb{Z}\) tali che \(n,m>0\) e sia il seguente sistema di congruenze:
        \[
            \begin{cases}
                x\equiv a\mod n\\
                x\equiv b\mod m\\
                x\in\mathbb{Z}
            \end{cases}
        \]
    \end{ipothesis}
    \begin{thesis}
        Il sistema ha soluzione se e solo se \((n,m)|b-a\), inoltre se \(c\) è una soluzione allora gli elementi di \([c]_{[n,m]}\) sono tutte e sole le soluzioni del sistema (le soluzioni in \(\mathbb{Z}\) sono: \( c+k[n,m]\in\Z\quad k\in\Z \)).
    \end{thesis}
    \begin{proof}
        Sia \(c\) una soluzione allora esistono \(h,k\in\Z\) tali che \(c=a+hn=b+km\) in quanto \(x\equiv a\mod n\) e \(x\equiv b\mod m\Rightarrow [x]_n=[a]_n\Rightarrow x=a+kn\), e \(x=b+km\), dunque \(a+kn=b+km\Rightarrow a-b=kn-km\Rightarrow a-b=k(n-m)\Rightarrow (n,m)|a-b\).
        Da prima possiamo dire che \(a-b = hn+km\) e che quindi \(a-hm=b+kn=c\). Si noti come \(c\) risolvi entrambe le equazioni.
        Sia ora \(S:=\{x\in\Z \mid x\text{ risolve il sistema}\}\) "l'insieme di tutte le soluzioni", bisogna provare che se \(c\in S\)a allora  \(S=[c]_{[n,m]}\) e inoltre cse \(c'\) è una soluzione allora \(c'\in[c]_{[n,m]}\).
        Quindi \(c=a+hn=b+km\quad c'=a+h'n=b+k'm\), dunque \(c-c'=h-h'n=k-k'm\), e quindi \(c-c'=h-h'n=k-k'm\), dunque sottraendo \(c\) a \(c'\) si ottiene che: \(c-c' = \cancel{a}+hn-\cancel{a}h'n=(h-h')n\) e \(c-c' = \cancel{b}+km-\cancel{b}k'm=(k-k')m\), dunque \(n|c-c'\) e \(m|c-c'\). Allora \([n,m] | c-c'\), e \(c'\equiv c\mod [n,m]\), dunque \(c'\in S\).
    \pushQED{}
    \end{proof}
    \raggedleft{{\ensuremath{\blacksquare}}}
\end{theorem}
\part{Teorema di Fermat-Eulero e Crittografia RSA}
\begin{lemma}
    Sia \(n\in\N\quad n\geq 2\).
    Allora \(n=p_1^{m_1}\cdot p_2^{m_2}\cdot\ldots\cdot p_k^{m_k}\) con \(p_i\) numeri primi a due a due distinti, allora vale che:
    \begin{itemize} 
        \item \(\phi(n)=\phi(p_1^{m_1})\cdot\phi(p_2^{m_2})\cdot\ldots\cdot\phi(p_k^{m_k})\), in quanto \(p_i\) è primo allora 
        \item \(\Rightarrow(p_1^{m}-p_1^{m-1})\cdot(p_2^{m}-p_2^{m-1})\cdot\ldots\cdot(p_k^{m}-p_k^{m-1})\). 
    \end{itemize}
\end{lemma}
\begin{lemma}
    Siano \(\alpha,\beta\in(\Z/_{n\Z})^*\), allora:
    \begin{itemize}
        \item \((\alpha\beta)^{-1}=\alpha^{-1}\beta^{-1}\)
        \item \((\alpha^{-1})^{-1}=\alpha\)
    \end{itemize}
    \begin{proof}
        Verifichiamo che la moltiplicazione tra le classi \(\alpha,\beta\) moltiplicata per le inverse esiste in quanto \(\alpha,\beta\in(\Z/_{n\Z})^*\), questa risulterà la classe di \([1]_n\) dunque che \((\alpha\beta)(\beta^{-1}\alpha^{-1})=[1]_n\).
        \begin{itemize}
            \item \((\alpha\beta)(\alpha^{-1}\beta^{-1})=\alpha(\beta\beta^{-1})\alpha^{-1}\) per la proprietà distributiva e associativa del prodotto in \((\Z/_{n\Z})^*\), dunque \((\alpha(\beta\beta^{-1})\alpha^{-1}=\alpha[1]_n\alpha^{-1}=\alpha\alpha^{-1}=[1]_n\), dunque \(\alpha\beta)^{-1}=\alpha^{-1}\beta^{-1}\) \checkmark.
            \item \((\alpha^{-1})^{-1}\alpha^{-1}=\alpha^{-1}(\alpha^{-1})^{-1}=[1]_n\), dunque \((\alpha^{-1})^{-1}=\alpha\).
        \end{itemize}
    \end{proof}
\end{lemma}
    \section{Teorema di Fermat-Eulero}
        \begin{theorem}
            Una qualsiasi classe invertibile elevata alla funzione di eulero è congruente alla classe unitaria.
            \begin{ipothesis}
                Sia \(n>0\).
            \end{ipothesis}
            \begin{thesis}
                allora \(\forall[\alpha]_n\in(\Z/_{n\Z})^*\Rightarrow[\alpha]^{\phi(n)}=[1]_n\).
                Notare come le classi prese in considerazione sono invertibili.
            \end{thesis}
            \begin{proof}
                Definiamo la funzione \(L_\alpha:(\Z/_{n\Z})^*\rightarrow(\Z/_{n\Z})^*\) definita come \(\beta\rightarrow\alpha\beta\). La presente è ben definitita per il lemma appena dimostrato, questa funzione è bigettiva, dimostriamo l'iniettività in quanto la surgettività ne sarà una conseguenza in quanto l'insieme di partenza e di arrivo sono uguali (vedi lemma cassetti).
                Supponiamo dunque, per assurdo, \(\exists \beta_1,\beta_2\in (\Z/_{n\Z})^*\) tali che \(L_\alpha(\beta_1)=L_\alpha(\beta_2)\Rightarrow \alpha\beta_1=\alpha\beta_2\rightarrow \beta_1=(\alpha\alpha^{-1})\beta_1=(\alpha^{-1})(\alpha\beta_1)\Leftrightarrow (\alpha^{-1})(\alpha\beta_2)=(\alpha^{-1}\alpha)\beta_2=\beta_2\), dunque \(\beta_1=\beta_2\), e quindi \(L_\alpha\) è iniettiva.
                Avendo dimostrato la bigettività di \(L_\alpha\) possiamo dire che \(L_\alpha(\beta_1)\ldots L_\alpha(\beta_k)=\alpha\beta_1\cdot\ldots\cdot\alpha\beta_k\) inoltre essendo il prodotto su \((\Z/_{n\Z})^*\) associativo possiamo dire che il precedente è \(=\alpha^k(\beta_1\ldots\beta_k)\), in quanto \(\beta_1,\ldots,\beta_k\) sono tutti e i soli elementi in \((\Z/_{n\Z})^*\) allora \(\beta_1\cdot\ldots\cdot\beta_k=\alpha^k(\beta_1\cdot\ldots\cdot\beta_k)\) moltiplicando a sinistra e destra per \(\beta_1^{-1}\cdot\ldots\cdot\beta_k^{-1}\) ottenimao che \([1]_n=\alpha^k\), in quanto come dimostrato \(k\) è in numero di classi in \((\Z/_{n\Z})^*\) e dunque \(k=\phi(n)\Rightarrow [1]_n=\alpha^{\phi(n)}\quad \forall\alpha\in(\Z/_{n\Z})^*\).
                \pushQED{}
            \end{proof}
            \raggedleft{{\ensuremath{\blacksquare}}}
        \end{theorem}
    \section{Teorema fondamentale della crittografia RSA}
        \begin{theorem}
            Una classe invertibile elevata ad un esponente \(d\) è congruente alla classe unitaria se e solo se \(d\) è l'inverso moltiplicativo di \(c\) modulo \(\phi(n)\).
            \begin{ipothesis}
                Sia \(c>0\) tale che: \((c,\phi(n))=1\) con \(n\) fissato \(>0\) e \(d>0: d\in [c]^{-1}_{\phi(n)}\).
            \end{ipothesis}
            \begin{thesis}
                Allora \(P_c\) è invertibile, e la sua inversa è \(P_c^{-1}=P_d\) dunque che \([d]_{\phi(n)}[c]_{\phi(n)}=[1]_{\phi(n)}\).
            \end{thesis}
            \begin{proof}
                Questo è equivalente a dire che: \(cd\equiv 1(\mod \phi(n))\Rightarrow \exists k\in\Z: cd=1+k\phi(n)\), applicando ora \(P_c,P_d\) su una \(\alpha\) classe otteniamo: \(P_d(P_c(\alpha))=(\alpha^c)^d=\alpha^{cd}=\alpha^{1+k\phi(n)}=\alpha\alpha^{k\phi(n)}=\alpha\), il che è verificato per le proprietà delle potenze e del prodotto in \((\Z/_{n\Z})^*\) e in quanto \(\alpha^{\phi(n)}=1\) per il teorema di eulero.
                Quindi questo dimostra che \(P_d(P_c(\alpha))=\alpha\), equivalentemente \([c]_{\phi(n)}[d]_{\phi(n)}=[1]_{\phi(n)}\) il che significa che \(P_c\) è invertibile e che la sua inversa è \(P_d\).
                \pushQED{}
            \end{proof}
            \raggedleft{{\ensuremath{\blacksquare}}}
        \end{theorem}
\pagebreak
\part{Teorema di equivalenza tra la congiungibilità con cammini e la congiungibilità con passeggiate e Teorema la relazione di congiungibilita è una relazione di equivalenza}
    \section{Teorema di equivalenza tra la congiungibilità con cammini e la congiungibilità con passeggiate}
        \begin{theorem}
            Due vertici di un grafo sono congiungibili per cammini se e solo se sono congiungibili per passeggiate.
            \begin{ipothesis}
                Supponendo di avere \(G=(V,E)\) un grafo e \(u,v\in V\) due vertici di \(G\).
            \end{ipothesis}
            \begin{thesis}
                Allora \(u\) è congiungibile con \(v\) in \(G\) per cammini se e solo \(\Leftrightarrow\) se \(u\) è congiungibile con \(v\) in \(G\) per passeggiate.
            \end{thesis}
            \begin{proof}
                \begin{enumerate}
                    \item[$\Rightarrow$] Banale, in quanto un cammino è una particolare passeggiata per definitzione.
                    \item[$\Leftarrow$] Supponiamo che \(u,v\) siano congiungibili in \(G\) per passeggiate, allora definitmo: \(\mathcal{P}:=\{P\mid\\ \text{P è una passeggiata da u a v}\}\) L'insieme delle passeggiate da \(u\) a \(v\) in \(G\). $\mathcal{A}:=\{n\in\N \mid \exists P\in\mathcal{P} : L(P)=n\}$ L'insieme delle lunghezze delle passeggiate da \(u\) a \(v\) in \(G\). 
                        In quanto \(\mathcal{A}\subseteq \N\) allora grazie al teorema del buon ordinamento di \(\N\) allora: 
                        \begin{align*}
                            \exists! \min(\mathcal{A})\Leftrightarrow & \exists P_0\in\mathcal{P}: L(P_0)=\min(\mathcal{A})=m \\
                            \Rightarrow & L(P_0)\leq L(P)\quad \forall P\in\mathcal{P}
                        \end{align*}
                        Dunque esiste un minimo dell'insieme \(\mathcal{A}\).
                        Dimostriamo ora per assurdo che \(P_0\) sia un cammino da \(v\) a \(w\):
                        Assumendo che \(P_0\) non sia un cammino da \(u\) a \(v\): \(P_0=\{v_0,\ldots,v_{i-1},v_i,v_{i+1},\ldots,v_{j-1},v_{j},v_{j+1},\ldots,v_{k})\}\), dunque \(\exists i,j\in\{0,1,\ldots,k\}\) tali che \(v_i=v_j\),
                        possiamo definire una passeggiata \(P_1\) tale che \(P_1:=\{u=v_0,\ldots,v_{i-1},v_i=v_j,v_{j+1},\ldots,v_{k=v})\}\), dunque \(L(P_1)=L(P_0)-(j-i)\), in quanto \(j-i\geq 1\) allora \(L(P_1)<L(P_0)\), il che lo renderebbe un minimo, ma questo è un assurdo in quanto va contro la definizione di \(P_0\) come minimo, dunque per assurdo \(P_0\) è un cammino da \(u\) a \(v\).
                    \end{enumerate}
                    \pushQED{}
            \end{proof}
            \raggedleft{{\ensuremath{\blacksquare}}}
        \end{theorem}
    \section{La relazione di congiungibilità è una relazione di equivalenza}
        \begin{theorem}
            La relazione di congiungibilità in un grafo è una relazione di equivalenza su \(V\).
            \begin{ipothesis}
                Dato \(G=(V,E)\) un grafo,
            \end{ipothesis}
            \begin{thesis}
                La relazione di congiungibilità in \(G\) su \(V\) è una relazione di equivalenza su \(V\) dunque:
                \begin{enumerate}
                    \item Riflessiva: \(u\sim u\quad \forall u\in V\)
                    \item Simmetrica: \(u\sim v\Rightarrow v\sim u\quad \forall u,v\in V\)
                    \item Transitiva: \(u\sim v\land v\sim w\Rightarrow u\sim w\quad \forall u,v,w\in V\)
                \end{enumerate}
            \end{thesis}
            \begin{proof}
                \begin{enumerate}
                    \item È verificana in quanto la passeggiata \(P=(u)\) è una passeggiata da \(u\) a \(u\).
                    \item È verificabile in quato se esiste una passeggiata \(P=(u=v_0,v_1,\ldots,v_k=v)\) possiamo definire un'altra passeggiata \(P'=(v=v_k,v_{k-1},\ldots,v_0=u)\) detta "inversa" che è una passeggiata da \(v\) a \(u\).
                    \item Supponendo ora che esistano le passeggiate \(P_0=(u=u_0,u_1,\ldots,u_k=v)\) e \(P_1=(v=v_0,v_1,\ldots,v_h=w)\) allora possiamo definire una passeggiata \(P_2=(u=v_0,v_1,\ldots,u_{k-1},u_{k}=v=v_{1},v_{2},\ldots,v_h=w)\) che è una passeggiata da \(u\) a \(w\).
                \end{enumerate}
                \pushQED{}
            \end{proof}
            \raggedleft{{\ensuremath{\blacksquare}}}
        \end{theorem}
\part{Teorema della relazione fondamentale tra il numero dei lati e i gradi dei vertici di un grafo finito e Lemma delle strette di mano}
    \section{Teorema della relazione fondamentale tra il numero dei lati e i gradi dei vertici di un grafo finito}
        \begin{theorem}
            \begin{ipothesis}
                Sia \(G=(V,E)\) un grafo finito.
            \end{ipothesis}
            \begin{thesis}
                Allora vale \(2|E|=\sum_{v\in V}\deg_G(v)\)
            \end{thesis}
            \begin{proof}
                Sia \(V=\{v_1,\ldots,v_n\}\) i vertici di \(G\) e \(E=\{e_1,\ldots,e_k\}\) i lati di \(G\) con \(k:=|E|\).
                Sia ora:\[
                    M_{ij}:=\begin{cases}
                        0& v_i\notin e_j\\
                        1& v_i\in e_j
                    \end{cases}\quad \forall i\in\{1,\ldots,n\},j\in\{1,\ldots,k\}
                \] ovvero la matrice di adiacenza del grafo \(G\).
                Allora:
                \[
                    \begin{aligned}
                        (1)\quad \sum_{j=1}^k\left(\sum_{i=1}^n M_{ij}\right)=&\sum_{i=1}^n\left(\overbrace{\left|\{e\in E\mid v_i\in e\}\right|}^{\text{def. }\deg_G(v)}\right)=\sum_{i=1}^n\deg_G(v_i)\\
                        (2)\quad \sum_{i=1}^n m_{ij}=&\left|\{ i\in \{1,\ldots,n\}\mid v_i\in e_j\}\right|=2\Rightarrow \sum_{i=1}^n\left(\sum_{j=1}^k M_{ij}\right)=\sum_{j=1}^k 2=2k=2|E|
                    \end{aligned}
                \]
                quindi per la proprietà commutativa della somma possiamo dire che:
                \[
                    \sum_{i=1}^n\sum_{j=1}^k m_{ij} =\sum_{j=1}^k\sum_{i=1}^n m_{ij}\overbrace{=}^{(1)}\sum_{i=1}^n\deg_G(v_i)\overbrace=^{(2)}2|E|
                \]
                \pushQED{}
            \end{proof}
            \raggedleft{{\ensuremath{\blacksquare}}}
        \end{theorem}
    \section{Lemma delle strette di mano}
        \begin{theorem}
            \begin{ipothesis}
                Sia \(G=(V,E)\) un grafo finito.
            \end{ipothesis}
            \begin{thesis}
                Il numero di vertici con grado dispari è pari.
            \end{thesis}
            \begin{proof}
                Definiamo \(D:=\{v\in V\mid 2\not|\deg_G(v)\}\) l'insieme dei vertici con grado dispari e \(P:=\{v\in V\mid 2|\deg_G(v)\}\) l'insieme dei vertici con grado pari \(\Rightarrow P\cap D=\emptyset\land P\cup D=V\).
                Grazie alla relazione dondamentale dei grafi finiti:
                \[
                    \begin{aligned}
                        2|E|&=\sum_{v\in V}\deg_G(v)=\sum_{v\in P}\deg_G(v)+\sum_{v\in D}\deg_G(v)\\
                        \sum_{v\in D}\deg_G(v)&=2|E|-\sum_{v\in P}\deg_G(v)
                    \end{aligned}
                \]
                In quanto \(\deg_G(v)\ \forall v\in P\) è pari allora \(\sum_{v\in P}\deg_G(v)\) è pari in quanto somma di numeri pari, inoltre \(2|E|\) è pari in quanto qualsiasi numero moltiplicato per un numero pari è pari. Allora \(\sum_{v\in D}\deg_G(v)\) è pari perchè è sottrazzione di pari, ma in quanto \(\deg_G(v)\ \forall v\in D\) è dispari allora la somma di questi è pari \(\Leftrightarrow\) il numero di vertici con grado dispari è pari.
                \pushQED{}
            \end{proof}
            \raggedleft{{\ensuremath{\blacksquare}}}
        \end{theorem}
\part{Teorema caratterizzante degli alberi finiti}
    \section{Teorema caratterizzante degli alberi finiti}
        \begin{theorem}
            \begin{ipothesis}
                Sia \(T=(V,E)\) un grafo finito
            \end{ipothesis}
            \begin{thesis}
                Allora le seguenti affermazioni sono equipotenti:
                \begin{enumerate}
                    \item \(T\) è un albero.
                    \item \(\forall v,v' \in V(T)\quad\exists!\) cammino da \(v\) a \(v'\).
                    \item \(T\) è connesso e \(\forall e\in E(T), T-e:=(V,E\setminus\{e\})\) non è connesso.
                    \item \(T\) non ha cicli e \(\forall e\in \binom{V}{2}\setminus E(T), T+e:=(V,E\cup\{e\})\) ha almeno un ciclo.
                    \item \(T\) è connesso e \(|E|=|V|-1\).
                \end{enumerate}
            \end{thesis}
            \begin{proof}
                \begin{enumerate}
                    \item[$1\Rightarrow 5$)] Si procede per induzione di prima forma su \(|V(T)|\) partendo da \(|V(T)|=1\).
                        \begin{base}[$|V(T)|=1$]
                            In questo caso \(T=(\{v\},\emptyset)\) è un albero in quanto un singolo vertice, e \(|E|=0,|V|=1\Rightarrow |E|=|V|-1=0\).
                            \checkmark\end{base}
                        \begin{step}[$|V(T)|\geq2\quad |V(T)-1|\Longrightarrow$]
                            In quanto \(|V(T)|\geq 2\) allora per il "Lemma delle foglie" questo ha almeno due foglie, sia dunque \(v\) una di queste, allora:
                            il grafo \(T-v\) è un albero per il lemma sopracittato. Vale che \(|V(T-v)|=|V(T)|-1\), e \(|E(T-v)|=|E(T)|-1\), in quanto \(v\) è una foglia (e quinidi \(\deg_T(v)=1\)) allora: \[
                                \begin{aligned}
                                    |V(T-v)|-1&=|E(T-v)|\\
                                    |V(T)|\cancel{-1}-1&=|E(T)|\cancel{-1}-1\\
                                    |V(T)|-1&=|E(T)|
                                \end{aligned}
                            \]
                            il che è verificato in quanto \(T-v\) è un albero e \(|V(T-v)|=|V(T)|-1\), dunque \(T\) è connesso e \(|E|=|V|-1\).
                        \end{step}
                        \popQED{}
                    \item[$1\Leftarrow 5$)] Si procede per induzione di prima forma su \(|V(T)|\) partendo da \(|V(T)|=1\).
                        \begin{base}[$|V(T)|=1$]
                            Per \(|V(T)|=1\) abbiamo che \(T=(\{v\},\emptyset)\) verifica \(|E|=|V|-1=0\) ed è un albero in quanto è un singolo vertice.
                        \end{base}
                        \begin{step}[$|V(T)|\geq2\quad |V(T)-1|\Longrightarrow |V(T)|$]
                            Sia \(T\) un grafo connesso che verifica la formula di Eulero \(|E|=|V|-1\) con \(|V(T)\).
                            Sapendo che \(T\) è connesso si dimostra per assurdo come questo abbia almeno una foglia: 
                            \begin{proof}
                                Supponiamo per assurdo che \(T\) non abbia foglie, quindi \(\forall v\in V(T)\ \deg_T(v)\geq 2\), inoltre sappiamo per ipotesi che l'equazione di eulero è verificata, dunque: \[
                                    \begin{aligned}
                                        2|E|&=\sum_{v\in V(T)}\deg_T(v)\\
                                        2(|V|-1)&=\sum_{v\in V(T)}\deg_T(v)\\
                                        2|V|-2&=\sum_{v\in V(T)}\deg_T(v)
                                        2|V|-2&\geq 2|V|\\
                                        -2&\geq 0
                                    \end{aligned}
                                \]
                                questo significa che se \(T\) non avesse foglie allora \(T\) non sarebbe connesso, il che è un assurdo, dunque \(T\) ha almeno una foglia.
                            \end{proof}
                            Prendiamo in considerazione ora una foglia \(v\in V(T)\), allora \(T-v\) è un grafo connesso in quanto \(T\) è connesso e \(v\) ne è una sua foglia, vale inoltre che: \[
                                \begin{aligned}
                                    |V(T-v)|&=|V(T)|-1\\
                                    |E(T-v)|&=|E(T)|-1\\
                                    |V(T-v)|-1&=|E(T-v)|\\
                                    |V(T)|-1-1&=|E(T)|-1\\
                                    |V(T)|-2&=|E(T)|-1\\
                                    |V(T)|-1&=|E(T)|
                                \end{aligned} 
                            \]
                            il che conferma l'ipotesi induttiva, dunque \(T\) è un albero in quanto connesso e \(T-v\) è un albero per ipotesi induttiva,
                            \popQED{}
                        \end{step}
                        \begin{proof}[]
                            Verifichiamo inoltre che \(T\) sia un albero, prendiamo per assurdo che \(\exists\) un ciclo \(c\) in \(T\), ogni vettore \(v_i\in c\) ha \(\deg_T(v_i)\geq 2\) altrimenti questo non potrebbe essere un ciclo, ma in quanto \(v\) è una foglia allora \(c\) è anche un ciclo in \(T-v\), il che và contro l'ipotesi induttiva, dunque \(T\) è un albero.
                        \end{proof}
                        Dunque in conclusione possiamo dire che il passo induttivo è stato fatto e che se il grafo \(T\) è connesso e \(|E|=|V|-1\) allora \(T\) è un albero.
                \end{enumerate}
                \pushQED{}
            \end{proof}
            \raggedleft{{\ensuremath{\blacksquare}}}
        \end{theorem}
\part{Teorema di esistenza dell'albero di copertuta per i grafi finiti}
    \section{Teorema di esistenza dell'albero di copertura per i grafi finiti}
        \begin{theorem}
            \begin{ipothesis}
                Sia \(G=(V,E)\) un grafo finito connesso.
            \end{ipothesis}
            \begin{thesis}
                G ammette almeno un albero di copertura.
            \end{thesis}
            \begin{proof}
                Definiamo il seguente insieme:\(e:=\{c:c\text{ è un sottografo connesso di G} \land V(c)=V(G)\}\) in quanto \(G\in e\) allora \(\Rightarrow E\neq \emptyset\). Definiamo inoltre \(S:=\{n\in\N: n=|E(C)| \text{ per qualche } c\in e\}\) l'insieme delle cardinalità degli insiemi di lati dei sottografi connessi di \(G\), anche questo non è vuoto \((S\neq \emptyset)\) in quanto \(|E(G)|\in S\), in quanto \(G\) è un sottografo connesso di \(G\).
                Dato che l'insieme \(S\subseteq \N\land S\neq \emptyset\) allora per il teorema del buon ordinamento di \(\N\) \(S\) ha un minimo, definiamo questo minimmo: \(\exists\bar{C}\in e: \min(S)=\left|V(\bar{C})\right|\).
                Per costruzuione \(V(\bar{C})=V(G)\), dimostriamo dunque per assurdo ce \(\bar{C}\) è un albero:
                Se \(\bar{C}\) non fosse un albero allora \(\exists e\in E(\bar{C}):\bar{C}-e\)  è connesso, dunque \(\left|E(\bar{C}-e)\right|=\left|E(\bar{C})\right|-1\), in quanto \(\bar{C}-e:=(V(\bar{C}),E(\bar{C}\setminus \{e\}))\), questo però comporta che \(\left|E(\bar{C}-e)\right|<\left|E(\bar{C})\right|\) il che è un assurdo in quanto và contro la definizione di \(\bar{C}\) come sottografo con \(\left|E(\bar{C})\right|=\min(S)\), dunque \(\bar{C}\) è un albero.
                \pushQED{}
            \end{proof}
            \raggedleft{{\ensuremath{\blacksquare}}}
        \end{theorem}
\end{document}